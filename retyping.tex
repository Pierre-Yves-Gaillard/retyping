%retyping-ah
% !TEX encoding = UTF-8 Unicode
\documentclass[12pt]{article}
\usepackage{amssymb,amsmath}
\usepackage[a4paper]{geometry}
\usepackage[T1]{fontenc}
\usepackage[utf8]{inputenc}
\usepackage[french]{babel}
\usepackage[pdfusetitle]{hyperref}
\usepackage{datetime}
\addtolength{\parskip}{.5\baselineskip}
%\newtheorem{theorem}{Theorem}
\newcommand{\nn}{\noindent}
\newcommand{\card}{\operatorname{card}}
\newcommand{\Top}{\operatorname{Top}}
\newcommand{\Ab}{\operatorname{Ab}}
\newcommand{\Hom}{\operatorname{Hom}}
\newcommand{\ZZ}{\mathbb Z}
\newcommand{\Unif}{\operatorname{Unif}}
\newcommand{\Ob}{\operatorname{Ob}}
\newcommand{\cat}{\operatorname{cat}}
\newcommand{\Ens}{\operatorname{Ens}}
\newcommand{\Ord}{\operatorname{Ord}}
\newcommand{\Gr}{\operatorname{Gr}}
\newcommand{\Fl}{\operatorname{Fl}}
\newcommand{\Hhom}{\mathcal H om}
\newcommand{\NN}{\mathbb N}
\title{Retyping a Text of Grothendieck}
\author{Pierre-Yves Gaillard}
\date{\today, \currenttime}
\begin{document}
\maketitle

\nn The text mentioned in the title is the Appendix, entitled \emph{Univers}, by N. Bourbaki to the SGA 4.1 Exposé by Grothendieck and Verdier, entitled \emph{Préfaisceaux}. The present text is a retyping of \emph{Univers}.

It seems to be an open secret that this appendix was written by Grothendieck. We shall admit this conjecture. (Note that the text contains several sentences written in the first person singular --- that is, with a "Je".)

This appendix is currently available in the form of 

$\bullet$ a scan of the original text [OV] (for "original version"),  

$\bullet$ a PDF file of a texified version [TV]. 

\nn(See details and links below.) [OV] is of a very poor typographical quality, whereas [TV], in its current form, is just unreadable. If you have a doubt about this, please see the part of [TV] which corresponds to page 195 of [OV]. (This is on page 112 of [TV]. The page breaks of [OV] are clearly indicated in [TV].) The differences between [OV], [TV] and this text are indicated in the footnotes.

The precise reference to the SGA Exposé is:

\nn Préfaisceaux, SGA 4.1, Grothendieck, A. and Verdier, J.-L. (1972). Préfaisceaux. In Artin, M., Grothendieck, A., and Verdier, J.-L., editors, Théorie des Topos et Cohomologie \'Etale des Schémas, volume~1 of Séminaire de géométrie algébrique du Bois-Marie, 4, pages 1-218. Springer-Verlag.

Here are some links:

\nn$\bullet$ Scans of the original text [OV]: 

\nn\href{http://library.msri.org/books/sga/}{http://library.msri.org/books/sga/}

\nn\href{http://modular.math.washington.edu/home/wstein/www/sga/}{http://modular.math.washington.edu/home/wstein/www/sga/} 

\nn$\bullet$ PDF file of the texified version [TV]:

\nn Indirect link (Stein): \href{http://goo.gl/IzUIXa}{http://goo.gl/IzUIXa} (source available via the above link).

\nn Direct link (Stein): \href{http://goo.gl/1fe3SN}{http://goo.gl/1fe3SN}

\nn Direct link (Laszlo):

\nn\href{http://www.math.polytechnique.fr/~laszlo/sga4/SGA4-1/sga41.pdf}{http://www.math.polytechnique.fr/$\sim$laszlo/sga4/SGA4-1/sga41.pdf}

\nn Direct link (Gaillard): \href{http://goo.gl/u8OEhO}{http://goo.gl/u8OEhO}

Here is Grothendieck's text:

\newpage

{\huge Appendice: Univers (par N. Bourbaki\footnote{Nous reproduisons ici, avec son accord, des papiers secrets de N. Bourbaki. Les références de ce texte se rapportent à son savant ouvrage. [This footnote is from Grothendieck; all the other footnotes are added by Gaillard.]})}

\tableofcontents

\section{Définition et premières propriétés des univers}

\nn\textbf{Définition~1.} \emph{Un ensemble $U$ est appelé un univers s'il satisfait aux conditions}:

\nn(U.I) \emph{si $x\in U$ et si $y\in x$, alors $y\in U$};

\nn(U.II) \emph{si $x$, $y\in U$, alors $\{x,y\}\in U$};

\nn(U.III) \emph{si $x\in U$, alors $\mathcal{P}(x)\in U$};

\nn(U.IV) \emph{si $(x_{\alpha})_{\alpha\in I}$ est une famille d'éléments de $U$, et si $I\in U$, alors la réunion}
$$
\bigcup_{\alpha\in I}x_{\alpha}
$$ 
\emph{appartient à $U$};

\nn(U.?) \emph{si $x$, $y\in U$, alors le couple\footnote{In [TV] "couple" is written "coupe". (Recall that [OV] and [TV] stand respectively for "original version" and "texified version". For more details, see page~1.)} $(x,y)\in U$.}

\nn\textbf{N.B.} Comme il a été je crois décidé pour les prochaines éditions, on définit le couple à la Kuratowaki par $(x,y)=\{\{x,y\}, \{x\}\}$, la condition (U.?) est inutile car elle résulte de (U.II).

\nn\textbf{Exemples.}

\nn1) L'ensemble \emph{vide} est un univers noté $U_0$.

\nn2) Considérons les mots finis non vides formés avec les quatre symboles 
$$
``\{",\quad''\}",\quad``,"\quad\text{et}\quad``\varnothing"
$$ 
(cf. Alg. I). Définissons, par récurrence sur la longueur $n$ d'un tel mot, la notion de mot \emph{significatif}: 

\nn(a) Pour $n=1$, seul le mot $\varnothing$ est significatif;

\nn(b) pour qu'un mot $A$ de longueur $n$ soit significatif, il faut et suffit qu'il existe $p$ mots significatifs distincts $A_1,\ldots,A_p$ $(p\geq 1)$ de longueurs $ < n$ tels que
$$
A=\{A_1,A_2,\ldots,A_p\}.
$$

\nn Par exemple $\varnothing$, $\{\varnothing\}$, $\{\{\varnothing\},\{\{\varnothing\}\},\{\varnothing,\{\varnothing\}\}\}$ sont des mots significatifs. Il est clair, par récurrence sur la longueur, que tout mot significatif désigne un terme de la Théorie des Ensembles; par exemple\footnote{In [TV] "par exemple" is written "par exemples".}, si $A=\{A_1,A_2,\ldots,A_p\}$, $A$ désigne l'ensemble dont les éléments sont $A_1$, $A_2,\ldots,$ et $A_p$; ces termes sont évidemment des \emph{ensembles finis}. Soit $U_1$ l'ensemble des ensembles ainsi obtenus. On vérifie aisément que $U_1$ satisfait aux conditions (U.I), (U.II), (U.III) et (U.IV) de la déf.~1 (mais non à cette idiote de (U.?), ce qui n'est pas grave si on veut bien décanuler le couple). Donc $U_1$ est un \emph{univers}. On notera que les éléments de $U_1$ sont finis, et que $U_1$ est dénombrable.

Dans les énoncés qui suivent, $U$ \emph{désigne un univers}.

\nn\textbf{Proposition~1.} \emph{Si $x\in U$\footnote{In [OV] and [TV] lower case $x$ is written upper case $X$.} et si $y\subset x$, alors $y\in U$.}

En effet, on a $\mathcal{P}(x)\in U$ par (U.III), d'où $y\in\mathcal{P}(x)$ et $y\in U$ par (U.I).

\nn \textbf{Corollaire.} \emph{Si $x\in U$, tout ensemble quotient $y$ de $x$ est élément de $U$.}

En effet $y$ est une partie de $\mathcal{P}(x)$. D'où $y\in U$ par (U.III) et la prop.~1.

\nn\textbf{Proposition~2.} \emph{Si $x\in U$, on a $\{x\}\in U$.}

Ça résulte de (U.II) appliqué pour $y=x$.

\nn\textbf{Proposition~3.} \emph{Tout couple, tout triplet, tout quadruplet d'éléments de $U$ est un élément de $U$.}

C'est vrai pour les couples d'après le N.B. ou (U.?). On en déduit le cas des triplets car $(x,y,z)=((x,y),z)$, puis celui des quadruplets car $(x,y,z,t)=((x,y,z),t)$.

\nn\textbf{Proposition 4.} \emph{Si $X$, $Y\in U$, alors $X\times Y\in U$.}

En effet, pour $x\in X$, $\{x\}\times Y$ est la réunion de la famille $\{(x,y)\}_{y\in Y}$; comme on a $\{(x,y)\}\in U$ d'après (U.I), (U.II) et la prop.3, on a $\{x\}\times Y\in U$ d'après (U.IV). Enfin $X\times Y$ est la réunion de la famille $(\{x\}\times Y)_{x\in X}$; c'est donc un élément de $U$ d'après (U.IV) encore.

\nn\textbf{Corollaire~1.} \emph{Si $X$, $Y$, $Z,\ldots$ sont des éléments de $U$, tous les ensembles de l'échelle construite sur $X$, $Y$, $Z,\ldots$ sont des éléments de $U$ (cf. chap.IV, $\S$\footnote{Mysterious reference...}).}

Ça résulte en effet d'applications successives de la prop.~4 et de (U.III).

\nn\textbf{Corollaire~2.} \emph{Si $(x_{\alpha})_{\alpha \in I}$ est une famille d'éléments de $U$ et si $I\in U$, l'ensemble somme $\sum_{\alpha \in I}x_{\alpha}$ est un élément de $U$.}

En effet, cet ensemble somme est une partie du produit 
$$
\left(\bigcup_{\alpha\in I} x_{\alpha}\right)\times I
$$ 
(chap. II\footnote{Mysterious reference...}), produit dont les deux facteurs sont éléments de $U$ (par (U.IV) et l'hypothèse). On applique alors les prop. 4 et 1.

\nn\textbf{Proposition 5.} \emph{Si $X$ et $Y$ sont des éléments de $U$, toute correspondance entre $X$ et $Y$ (en particulier toute application de $X$ dans $Y$) est un éléments de $U$.}

En effet, une telle correspondance $C$ est un triplet $(X,Y,\Gamma)$ où $\Gamma$ est une partie de $X\times Y$ (le graphe de $C$) (chap. II, \S). On a $\Gamma\in U$ d'après les prop. 4 et 1. D'où $C\in U$ par la prop.~3.

\nn\textbf{Proposition 6.} \emph{Si $X$, $Y\in U$, tout ensemble $Z$ de correspondances entre $X$ et $Y$ (en particulier d'applications de $X$ dans $Y$) est élément de $U$.}

En effet, $Z$ est une partie de $\{X\}\times\{Y\}\times\mathcal{P}(X\times Y)$. Or ce produit est élément de $U$ d'après la prop.~2 et le cor. à la prop.~4. On a donc $Z\in U$ par la prop.~1.

\nn\textbf{Corollaire.} \emph{Si $(x_{\alpha})_{\alpha \in I}$ est une famille d'éléments de $U$, et si $I\in U$, on a $\Pi_{\alpha\in I}x_{\alpha} \in U$.}

En effet, ce produit est un ensemble d'applications de $I$ dans $\bigcup_{\alpha\in I}x_{\alpha}$, et cette réunion est élément de $U$ par (U.IV).

\nn\textbf{Proposition 7.} \emph{Si $X$ est une partie de $U$ dont le cardinal est au plus celui d'un élément de $U$, alors $X$ est un élément de $U$.}

Soit $I$ un élément de $U$ tel que $\card(X)\leq \card(I)$. On a une surjection $i\mapsto x_i$ d'une partie $I'$ de $I$ sur $X$. Alors $X$ est la réunion de la famille $(\{x_i\})_{i\in I'}$; or cette réunion est élément de $U$ par la prop.~1, la prop.~2 et (U.IV).

\nn\textbf{Corollaire.} \emph{Si $U$ est non vide toute partie finie de $U$ est un élément de $U$, et $U$ a des éléments de cardinal fini arbitraire.}

Par contre, si $U=\varnothing$, $\varnothing$ est une partie finie de $\varnothing$ mais non un élément de $\varnothing$.

En effet, si $U$ est non vide, la prop.~2 montre qu'un ensemble $x_0$ à un élément appartient à $U$. Par récurrence sur $n$, posons $x_{n+1}=\mathcal{P}(x_n)$. On a $x_n\in U$ par (U.III), et $\card(x_{n+1})=2^{\card(x_n)}$, de sorte que $U$ a des éléments de cardinal fini arbitrairement grand.

\nn\textbf{Remarque.} Il résulte de la prop.~2 et du cor. à la prop.~7 que tout univers non vide $U$ contient l'univers $U_1$ de l'ex.~2; en effet on a $\varnothing\in U$ d'après la prop.~1. Ainsi $U_1$ est l'intersection de tout les univers non vides. Plus généralement:

\nn\textbf{Proposition 8.} \emph{Si $(U_{\lambda})_{\lambda\in L}$ est une famille non vide d'univers, alors $U=\bigcap_{\lambda\in L}U_{\lambda}$ est un univers.}

Ceci résulte aussitôt de la déf.~1.

\section{Univers et espèces de structures}

Soit $\mathcal{E}$ une espèce de structure; supposons, pour fixer les idées et alléger l'exposé, que chaque structure d'espèce $\mathcal{E}$ est définie sur un ensemble de base. Soit $(X,S)$ une structure d'espèce $\mathcal{E}$ ($X$ étant l'ensemble de base, et $S$ la structure), et soit $U$ un \emph{univers}. Si $X\in U$, alors tous les objets constitutifs de la structure $S$ sur $X$ sont éléments de $U$ (par le cor.~1 de la prop.~4, $\rm n^\circ$~1 et par (U.I)) de sorte que la structure $(X,S)$ est élément de $U$.

Supposons maintenant que $\mathcal{E}$ soit une espèce de structure avec \emph{morphismes}. $X$ et $X'$ sont des éléments de $U$ munis de structures d'espèce $\mathcal{E}$, alors l'ensemble des morphismes de $X$ dans $X'$ est encore un élément de $U$\footnote{Mysterious reference...} ($\rm n^\circ$~1, prop.~6).

Considérons alors la \emph{catégorie} $(U\text{-}\mathcal{E})$ définie au $\S1,\rm n^\circ$~2, ex. c)\footnote{In [TV] it is written ``du $U$'' instead of ``de $U$''.}. Comme les objets et les flèches de $(U\text{-}\mathcal{E})$ sont des éléments de $U$, $(U\text{-}\mathcal{E})$ est un \emph{couple de parties} de $U$, muni d'un quadruplet d'applications; ce n'est pas, en général, un élément de $U$. La notation $X\in (U\text{-}\mathcal{E})$ voudra dire que $X$ est un élément de $U$ muni d'une structure d'espèce $\mathcal{E}$.

La catégorie $(U\text{-}\mathcal{E})$ est \emph{stable} pour de nombreuses opérations:

\nn a) Si $X\in(U\text{-}\mathcal{E})$, et si $X'$ est une partie de $X$ qui admet une \emph{structure induite}, alors $X'\in(U\text{-}\mathcal{E})$. Ça résulte de la prop.~1, $\rm n^\circ$~1.

\nn b) Si $X\in (U\text{-}\mathcal{E})$, et si $X''$ est un ensemble quotient de $X$ qui admet une \emph{structure quotient}, alors $X''\in
(U\text{-}\mathcal{E})$ (cor. de la prop.~1, $\rm n^\circ$~1).

\nn c) Si $X,Y\in (U\text{-}\mathcal{E})$, et si $X\times Y$ admet une \emph{structure produit}, alors $X\times Y\in(U\text{-}\mathcal{E})$ (prop.~4, $\rm n^\circ$~1). Plus généralement, si $(X_i)_{i\in I}$ est une famille d'éléments de $(U\text{-}\mathcal{E})$, \emph{si on a} $I\in U$, et si $\Pi_{i\in I}X_i$ admet une \emph{structure produit}, alors $\Pi_{i\in I}X_i\in (U\text{-}\mathcal{E})$ (cor. de la prop.~6). Assertions analogues pour les \emph{structures sommes} (cor.~2 de la prop.~4).

\nn d) Soient $I$ un ensemble préordonné, et $(X_i, f_{ij})_{i,j\in I}$ un \emph{système projectif} (resp \emph{inductif}) d'ensembles munis de structures d'espèce $\mathcal{E}$ et de morphismes; soit $L$ la limite de ce système, au sens du chap. III. Si $X_i\in U$ pour tout $i$, si $I\in U$, et si $L$ admet une structure limite projective (resp. inductive), alors on a $L\in(U\text{-}\mathcal{E})$: en effet $L$ est une partie de $\Pi_{i\in I}X_i$ (resp. un ensemble quotient de $\sum_{i\in I}X_i$), et est donc un élément de $U$ d'après le $\rm n^\circ$1.

Donnons encore quelques exemples plus particuliers:

\nn${}^\bigstar$1) Soient $X$, $Y\in (U\text{-}\Top)$ deux espaces \emph{localement compacts}. Alors l'ensemble $\mathcal{C}(X,Y)$ des applications continues de $X$ dans $Y$, muni de la topologie de la convergence compacte (Top. Géné. $X$), est un élément de $(U\text{-}\Top)$. En effet on a $\mathcal{C}(X,Y)\in U$ d'après la prop.~6 du $\rm n^\circ\ 1.{}_\bigstar$

\nn${}^\bigstar$2) Soient $X$, $Y\in (U\text{-}\Ab)$. Alors le groupe $\Hom_{\ZZ}(X,Y)$ est un élément de $(U\text{-}\Ab)$ par la prop.~6 du $\rm n^\circ$~1. Si on suppose de plus qu'on a $\ZZ\in U$, le produit tensoriel $X\otimes_{\ZZ}Y$ est un élément de $(U\text{-}\Ab)$; en effet ce produit tensoriel est un quotient de $\ZZ^{X\times Y}$, lequel est une partie de $\ZZ^{X\times Y}$, qui lui-même est un élément de $U$ par les prop. 4 et 6 du $\rm n^\circ\ 1.{}_\bigstar$

\nn${}^\bigstar$3) Soit $X\in (U\text{-}\Unif)$ un espace uniforme. Alors son \emph{complété} $\widehat{X}$ est un élément de $(U\text{-}\Unif)$. En effet $\widehat{X}$ est un ensemble de classes d'équivalences de filtres de Cauchy sur $X$; or un filtre sur $X$ est un élément de $\mathcal{P}(\mathcal{P}(X))$, donc une classe d'équivalence de filtres est un élément de $\mathcal{P}(\mathcal{P}(\mathcal{P}(X)))$, de sorte que $\widehat{X}$ est un élément de $\mathcal{P}(\mathcal{P}(\mathcal{P}(\mathcal{P}(X))))$, donc un élément de $U$ par (U.III) et (U.I).${}_\bigstar$

\nn\textbf{N.B.} Moralité, Bourbaki devra veiller à bien \og canonifier\fg\ ses constructions. Par exemple, dans celles où on adjoint un élément $\infty$ (compactifié d'Alexandroff, corps projectif), il y aura intérêt à prendre $\infty =\varnothing$ (car $\varnothing$ est élément de tout univers non vide) et à former l'ensemble somme de $\{\varnothing\}$ et de l'ensemble donné. Bien entendu il faudrait aussi \og canonifier\fg\ l'ensemble à deux éléments servant à construire les ensembles sommes; ainsi $\{\varnothing,\{\varnothing\}\}$ me paraît un bon candidat car il appartient à tout univers non vide.

\section{Univers et catégories}

Soient $U$ un univers et $C$ une catégorie. Nous écrivons $C\in U$ si on a $\Ob(C)\in U$ et $\Fl(C)\in U$.  Cette écriture est justifiée du fait que $C$ est le sextuplet formé de $\Ob(C)$, de $\Fl(C)$ et des quatre applications structurales; donc si $\Ob(C)\in U$ et $\Fl(C)\in U$, ces quatre applications sont éléments de $U$ ($\rm n^\circ$~1, cor.~1 de la prop.~4), donc aussi le sextuplet $C$ ($\rm n^\circ$~1, prop.~3, \emph{cum grano salis}). Ceci est d'ailleurs un cas particulier du $\rm n^\circ$~2, si l'on considère l'espèce de structure \og$\cat$\fg\ de catégorie. Avec les notations du $\rm n^\circ$~2, la relation $C\in U$ s'écrit aussi $C\in (U\text{-}\cat)$.

On notera qu'une catégorie comme $(U\text{-}\Ens)$, $(U\text{-}\Ord)$, $(U\text{-}\Top)$ ou $(U\text{-}\Gr)$ n'est \emph{pas}, en général, un élément de $U$.

Soient $C$, $D$ deux catégories et $U$ un univers tels que $C$, $D\in U$. Si $C'$ est une \emph{sous-catégorie} de $C$, on a $C'\in U$; la \emph{catégorie produit} $C\times D$, la catégorie somme de $C$ et $D$, les catégories opposées $C^{\circ}$ et $D^{\circ}$ sont aussi éléments de $U$ (cf. $\rm n^\circ$~2). La \emph{catégorie de foncteurs} $E=\Hhom(C,D)$ est également un élément de $U$: en effet $\Ob(E)$ est un ensemble de couples d'applications $\Ob(C)\rightarrow\Ob(D)$, $\Fl(C)\rightarrow\Fl(D)$, donc $\Ob(E)\in U$ par le $\rm n^\circ$~1; quant à $\Fl(E)$, c'est un ensemble de morphisme fonctoriels, c'est-à-dire d'applications $\Ob(C)\rightarrow\Fl(D)$, d'où $\Fl(E)\in U$ par la prop.~6 du $\rm n^\circ$~1.

\nn\textbf{Proposition 9.} \emph{Soient $\mathcal{E}$ une espèce de structure avec morphismes, et $U$, $U'$ deux univers tels que $U'\subset U$. Alors $(U'\text{-}\mathcal{E})$ est une sous-catégorie pleine de $(U\text{-}\mathcal{E})$.}

En effet, si $x$ et $y$ sont des objets de $(U'\text{-}\mathcal{E})$, ce sont par définition des objets de $(U\text{-}\mathcal{E})$. Quant à $\Hom(x,y)$, c'est la même ensemble dans les deux catégories (cf. \S1, $\rm n^\circ$~2, ex.~c)).

\nn\textbf{N.B.} On notera que l'hypothèse que $U$ et $U'$ sont des univers  est inutile, de sorte que la prop.~9 remonterait avantageusement au $\S1, \rm n^\circ$~4. Mais la Tribu a demandé qu'elle soit ici.

\nn\textbf{Remarque.} Il arrive que les axiomes de l'espèce de structure $\mathcal{E}$ impliquent que les \emph{cardinaux} des ensembles munis de structures d'espèce $\mathcal{E}$ soient \emph{bornés} par un cardinal fixe, soit $\mathbf c$ (${}^\bigstar$par exemple l'espèce de structure de groupe fini, de groupe de type fini, de module de type fini sur un anneau fixe $A$, ou d'algèbre de type fini sur $A_\bigstar$). Supposons alors qu'il existe un élément $z$ de $U'$ tel que $\mathbf c\leq\card(z)$. Alors, pour tout ensemble $x$ muni d'une structure d'espèce $\mathcal{E}$, il existe un élément $x'$ de $U'$ équipotent à $x$, par exemple une partie de $z$; munissons $x'$ de la structure déduite de celle de $x$ par transport de structure; on obtient ainsi un élément de $(U'\text{-}\mathcal{E})$. Il s'ensuit que \emph{le foncteur d'inclusion de $(U'\text{-}\mathcal{E})$ dans $(U\text{-}\mathcal{E})$ est alors essentiellement surjectif} ($\S 4, \rm n^\circ$~2, déf.~2), \emph{et est donc une équivalence de catégorie} ($\S 4, \rm n^\circ$~3, th.~1).

\section{L'axiome des univers}

Les fort agréables résultats de stabilité du $\rm n^\circ$~2 n'ont d'intérêt que si on peut les appliquer à autre chose qu'aux deux petits univers $U_0$ et $U_1$ décrits au ${\rm n^\circ}$~1. Nous ajouterons donc aux axiomes de la Théorie des Ensembles l'axiome suivant:

(A.6) \emph{Pour tout ensemble $x$, il existe un univers $U$ tel que $x\in U$.}

Cet axiome implique que, si $(x_{\alpha})_{\alpha\in I}$ est une \emph{famille} d'ensembles, il existe un univers $U$ tel que $x_{\alpha}\in U$ pour tout $\alpha\in I$: il suffit, en effet, d'appliquer (A.6) à $x=\bigcup_{\alpha\in I}x_{\alpha}$ et d'appliquer la prop.~1 du $\rm n^\circ$~1.

En particulier, étant donnée une \emph{catégorie} $C$, il existe un univers $U$ tel que $C\in U$ au sens du $\rm n^\circ$~3: on applique ce qui précède à la famille $(\Ob(C),\Fl(C))$. Ceci s'applique aux catégories de la forme $(V\text{-}\mathcal{E})$ où $\mathcal{E}$ est une espèce de structure avec morphismes et $V$ un ensemble, un univers par exemple; en général on a $V\neq U$.

Par exemple, si $V$ est un univers, on a $\Ob(V\text{-}\Ens)=V$ et $\Fl(V\text{-}\Ens)\subset V$ ($\rm n^\circ$~1, prop.~5). Les univers $U$ tels que $(V\text{-}\Ens)\in U$ sont donc ceux tels que $V\in U$. Or la relation $V\in V$ est impossible pour un univers: en effet, pour toute partie $A$ de $V$, on aurait $A\in V$ ($\rm n^\circ$~1, prop.~1), d'où $\card(\mathcal{P}(V))\leq\card(V)$, ce qui est impossible.

\section{Univers et cardinaux fortement inaccessibles}

Soit $U$ un univers. D'après la condition (U.I) de la déf.~1 tout élément $x$ de $U$ est une partie de $U$; on a donc $\card(x)\leq\card(U)$. Comme les cardinaux des parties de $U$ forment un ensemble bien ordonné, le cardinal
\begin{equation*}
\mathbf c(U)={\sup}_{x\in U}\card(x)\tag{1}
\end{equation*}
existe. Notons que, pour tout cardinal $\mathbf c < \mathbf c(U)$, il existe un élément $x$ de $U$ tel que $\card(x)=\mathbf c$; en effet, il existe par définition $y\in U$ tel que $\mathbf c\leq \card(y)\leq\mathbf c(U)$, et l'on prend pour $x$ une partie convenable de $y$. Réciproquement, si $x\in U$, on a $\card(x) < \mathbf c(U)$; en effet on a $\mathcal{P}(x)\in U$ par (U.III), d'où $2^{\card(x)}\leq\mathbf c(U)$. Le cardinal $\mathbf c(U)$ a donc les propriétés suivantes:

\nn1) Si $\mathbf c$ est un cardinal $ < \mathbf c(U)$, on a $2^{\mathbf c} < \mathbf c(U)$; en effet, si $x$ est un élément de $U$ de cardinal $\mathbf c$, on a $\mathcal{P}(x)\in U$ par application de (U.III), d'où $2^{\mathbf c} < \mathbf c(U)$.

\nn2) Si $(\mathbf c_{\lambda})_{\lambda\in I}$ est une famille de cardinaux telle que $\mathbf c_{\lambda} < \mathbf c(U)$ pour tout $\lambda\in I$ et que $\card(I) < \mathbf c(U)$, alors le cardinal somme $\sum_{\lambda \in I}\mathbf c_{\lambda}$ est $ < \mathbf c(U)$. En effet soit $x_{\lambda}\in U$ tel que $\card(x_{\lambda})=\mathbf c_{\lambda}$; quitte à remplacer $I$ par un ensemble d'indices équipotent, on peut supposer qu'on a $I\in U$; alors l'ensemble somme des $x_{\lambda}$ est élément de $U$ (cor.~2 de la prop.~4 du $\rm n^\circ$~1), ce qui démontre notre assertion.

Posons la définition suivante:

\nn\textbf{Définition~2.} Un cardinal $\mathbf d$ est dit fortement inaccessible si:

\nn $\bullet$ (FI.1) Si $\mathbf c$ est un cardinal tel que $\mathbf c < \mathbf d$, on a $2^{\mathbf c} < \mathbf d$.

\nn $\bullet$ (FI.2) Si $(\mathbf c_{\lambda})_{\lambda\in I}$ est une famille de cardinaux telle que $\mathbf c_{\lambda} < \mathbf d$ pour tout $\lambda\in I$ et si $\card(I) < d$, alors $\sum_{\lambda\in I}\mathbf c_{\lambda} < \mathbf d$.

\nn\textbf{Exemples.} Le cardinal 0 et le cardinal infini dénombrable sont fortement inaccessibles. Aucun cardinal fini non nul n'est fortement inaccessible. Ainsi nous venons de démontrer que l'axiome (A.6) des univers implique la relation:

(A'.6) \emph{Tout cardinal est strictement majoré par un cardinal fortement inaccessible.}

Inversement:

\nn\textbf{Théorème~1.} \emph{La relation} (A'.6) \emph{implique l'axiome} (A.6) \emph{des univers.}

En effet soit $A$ un ensemble. Il s'agit de construire un univers $U$ dont $A$ est élément. Définissons par récurrence une suite $(A_n)_{n\geq 0}$ d'ensembles au moyen de:
\begin{equation*}
A_0=A,\ A_{n+1}=\text{ réunion des éléments de $A_n=$ ensemble des éléments de $A_n$.}\tag{2}
\end{equation*}

Posons $B=\bigcup^\infty_{n=0} A_n$. Soit, par $(A'.6)$, $\mathbf c$ un cardinal fortement inaccessible tel que $\card(B) < \mathbf c$.

Il existe un ensemble \emph{bien ordonné} $I$ tel que $\card(I)=\mathbf c$. Quitte à remplacer $I$ par son plus petit segment de cardinal $\mathbf c$, on peut supposer que \emph{tout segment de $I$ distinct de $I$ a un cardinal} $ < \mathbf c$. Nous noterons $\mathcal{E}$ le \emph{plus petit élément de} $I$. Il résulte de l'hypothèse sur les segments que $I$ n'a pas de plus grand élément (sinon on l'enlèverait), donc que tout élément de $I$ admet un successeur;
pour $\alpha \in I$, nous noterons $s(\alpha)$ le \emph{successeur} de $\alpha$. 

Ceci étant, définissons, par \emph{récurrence transfinie}, une famille $(B_{\alpha})_{\alpha\in I}$ d'ensembles au moyen de:
\begin{equation*}
\begin{cases}& B_{\mathcal{E}}=B\\ & B_{s(\alpha)}=B_{\alpha}\cup \mathcal{P}(B_{\alpha})\\ & B_{\alpha}={\displaystyle{\bigcup_{\beta < \alpha}}}B_{\beta}\text{ si $\alpha$ n'a pas de prédécesseur.}
\end{cases}\tag{3}
\end{equation*}

Posons $U=\bigcup_{\alpha\in I}B_{\alpha}$. Nous allons montrer que $U$ est l'univers cherché.

On a d'abord $A\in U$, car $A=A_0$ est une partie de $B=B_{\mathcal{E}}$, donc un élément de $\mathcal{P}(B_{\mathcal{E}})\subset B_{s(\mathcal{E})}$.

Notons ensuite que, pour tout $\alpha \in I$, on a:
\begin{equation*}
\card(B_{\alpha}) < \mathbf c.\tag{4}
\end{equation*}
Procédons en effet par récurrence transfinie sur $\alpha$. C'est vrai pour $\alpha=\mathcal{E}$ par hypothèse. De (4) on déduit 
\begin{align*}
\card(B_{s(\alpha)})\leq\card(B_{\alpha})+2^{\card(B_\alpha)}&<\mathbf c+\mathbf c&\text{(par (FI.1))}\\ 
&=\mathbf c&\text{(car $\mathbf c$ est infini).}
\end{align*}
Enfin, si $\alpha$ n'a pas de prédécesseur, l'ensemble $I'$ des $\beta < \alpha$ a un cardinal $ < \mathbf c$ par construction; d'où (si $\card(B_{\beta}) < \mathbf c$ pour tout $\beta < \alpha$), 
$$
\card(B_{\alpha})=\card\left(\bigcup_{\beta\in I'}B_{\beta}\right) \leq \sum_{\beta\in
I'}\card(B_{\beta}) < \mathbf c
$$ 
(par (FI.2)). Ceci démontre (4).

Montrons maintenant qu'on a
\begin{equation*}
\card(x) < \mathbf c\text{ pour tout } x\in U.\tag{5}
\end{equation*}
En effet, il suffit de montrer que, pour tout $\alpha\in I$, on a \og$\card(x) < \mathbf c$ pour tout $x\in B_{\alpha}$\fg. Procédons encore par récurrence transfinie sur $\alpha$. C'est vrai pour $\alpha=\mathcal{E}$, car, si $x\in B_{\mathcal{E}}=B$, il existe $n\geq 0$ tel que $x\in A_n$; d'où $X \subset A_{n+1}$ par (2), $x\subset B$ et $\card(x) \leq \card(B) < \mathbf c$. Le cas où $\alpha$ n'a pas de prédécesseur est évident. Enfin, si $x\in B_{\alpha}$ et si $\alpha=s(\beta)$, on a soit $x\in B_{\beta}$ et l'assertion \og$\card(x) < \mathbf c$\fg\ est vraie par récurrence, soit $x\in\mathcal{P}(B_{\beta})$, d'où $x\subset B_{\beta}$ et l'assertion \og$\card(x) < \mathbf c$\fg\ est vraie par (4).

Nous sommes maintenant en mesure de démontrer que $U$ est bien un \emph{univers}:

(U.I) Soient $x\in U$ et $y\in x$. Il s'agit de montrer qu'on a $y\in U$. Autrement dit il s'agit de montrer que, pour tout $\alpha\in I$, on a la relation:
$$
``x\in B_{\alpha}\text{ et } y\in x\Longrightarrow y\in B_{\alpha}\text{''}.
$$
On procède encore par récurrence transfinie sur $\alpha$. C'est vrai pour $\alpha =\mathcal{E}$ car $x\in B_{\mathcal{E}}=B$ implique $x\in A_n$ pour un certain $n$; donc, si $y\in x$, on a $y\in A_{n+1}$, d'où $y\in B=B_{\mathcal{E}}$. Le passage à un élément $\alpha$ sans prédécesseur est évident. Passons enfin de $\alpha$ à $s(\alpha)$: si $x\in B_{s(\alpha)}$ et si $y\in x$, on a, soit $x\in B_{\alpha}$, d'où $y\in B_{\alpha}\subset B_{s(\alpha)}$ par récurrence, soit $x\in \mathcal{P}(B_{\alpha})$, d'où encore $y\in B_{\alpha}\subset B_{s(\alpha)}$.

(U.II) Soient $x$, $y\in U$. Comme la famille $(B_{\alpha})_{\alpha\in I}$ est croissante par (3), il existe $\alpha\in I$ tel que $x$, $y\in B_{\alpha}$. Alors $\{x,y\}\in\mathcal{P}(B_{\alpha})\subset B_{s(\alpha)}\subset U$.

(U.III) Soit $x\in U$. Il existe alors $\alpha\in I$ tel que $x\in B_{\alpha}$. Il suffit donc de montrer que, pour tout $\alpha\in I$, on a la relation:
$$
\text{\og}x\in B_{\alpha}\Longrightarrow \mathcal{P}(x)\in B_{s(s(\alpha))}\text{\fg}.
$$
On procède encore par récurrence transfinie sur $\alpha$. Si $\alpha=\mathcal{E}$, on a $x\in A_n$ pour un certain $n$, d'où $y\subset A_{n+1}\subset B=B_{\mathcal{E}}$ pour toute partie $y$ de $x$; ainsi on a $y\in \mathcal{P}(B_{\mathcal{E}})\subset B_{s(\mathcal{E})}$ pour tout $y\in \mathcal{P}(x)$, d'où $\mathcal{P}(x)\subset B_{s(\mathcal{E})}$ et $\mathcal{P}(x)\in\mathcal{P}(B_{s(\mathcal{E})})\subset B_{s(s(\mathcal{E}))}$, de sorte que notre assertion est vraie pour $\alpha= \mathcal{E}$. Le passage à un élément $\alpha$ sans prédécesseur est immédiat, car $\beta < \alpha$ implique alors $s(s(\beta)) < \alpha$. Passons enfin de $\alpha$ à $s(\alpha)$; soit $x\in B_{s(\alpha)}$; si $x\in B_{\alpha}$, on a $\mathcal{P}(x)\in B_{s(s(\alpha))}\subset B_{s(s(s(\alpha)))}$ par récurrence; si $x\in\mathcal{P}(B_{\alpha})$, on a $x\subset B_{\alpha}$, d'où $\mathcal{P}(x)\subset \mathcal{P}(B_{\alpha})$ et $\mathcal{P}(x)\in \mathcal{P}(\mathcal{P}(B_{\alpha}))\in B_{s(s(s(\alpha)))}$.

(U.IV) Soit $(x_{\lambda})_{\lambda\in K}$ une famille d'éléments de $U$ telle que $K\in U$. Il s'agit de montrer que la réunion $x=\bigcup_{\lambda\in K}x_{\lambda}$ est élément de $U$. Pour tout $\lambda\in K$, choisissons un $\alpha(\lambda)\in I$ tel que $x_{\lambda}\in B_{\alpha(\lambda)}$. Montrons que l'ensemble $\alpha(K)$ des $\alpha(\lambda)$ est \emph{majoré} dans $I$: en effet, s'il ne l'était pas, on aurait:
$$
I=\bigcup_{\lambda\in K}[\mathcal{E}, \alpha(\lambda)],
$$
ce qui contredirait (FI.2) car $\card(K) < \mathbf c$ et $\card([\mathcal{E},\alpha(\lambda)] < \mathbf c$ pour tout $\lambda\in K$. Soit donc $\beta$ un majorant de $\alpha(K)$; on a $x_{\lambda}\in B_{\beta}$ pour tout $\lambda\in K$ car la famille $(B_{\alpha})_{\alpha \in I}$ est croissante. On a donc $x_{\lambda}\in B_{\beta}$ d'après ce qu'on a vu dans la démonstration de (U.I), d'où $x=\bigcup_{\lambda\in K}x_{\lambda}\subset B_{\beta}$. Par (3) il s'ensuit qu'on a $x\in B_{s(\beta)}$, d'où $x\in U$. cqfd

\nn\textbf{Définition 3.} \emph{Soient $x$, $y$ deux ensembles et $n$ un entier $\geq0$. On dit que $y$ est un composant d'ordre $n$ de $x$ s'il existe une suite $(x_j)_{j=0,\ldots,n}$ telle que $x_0=x$, $x_n=y$, et $x_{j+1}\in x_j$ pour $j=0,\ldots,n-1$.}

Ainsi $x$ est le seul composant d'ordre 0 de $x$. Les composants d'ordre~1 (resp.~2) de $x$ sont les éléments de $x$ (resp. les éléments des éléments de $x$). On dit que $y$ est un \emph{composant} de $x$ s'il existe $n\geq 0$ tel que $y$ soit composant d'ordre $n$ de $x$. La relation \og$y$ est un composant de $x$\fg\ est une \emph{relation de préordre}. D'après le schéma de sélection-réunion (chap. II, \ldots) la relation \og$y$ est un composant de $x$\fg\ est collectivisante par rapport à $y$, de sorte que les composants de $x$ forment un \emph{ensemble}.

\nn\textbf{Définition 4.} \emph{Soit $\mathbf c$ un cardinal. On dit qu'un ensemble $x$ est de type $\mathbf c$ (resp. de type strict $\mathbf c$, de type fini) si tout les composants de $x$ ont des cardinaux $\leq\mathbf c$ (resp. $ < \mathbf c$, finis).}

\nn\textbf{Exemples.} Les éléments de l'univers $U_1$ ($\rm n^\circ$~1, ex.~2) sont tous de type fini. Si $\mathbf c$ est un cardinal, et si $x$ est de type $\mathbf c$ (resp. type strict $\mathbf c$, type fini), alors tout composant de $x$ et toute partie de $x$ sont de type $\mathbf c$ (resp. type strict $\mathbf c$, type fini); de même $\mathcal{P}(x)$ est de type $2^{\mathbf c}$ (resp. de type strict $2^{\mathbf c}$, de type fini). Si $x$ est de type
$\mathbf c$ et si $\mathbf c\leq\mathbf c'$, alors $x$ est de type $\mathbf c'$.

\nn\textbf{Lemme.} \emph{Soient $\mathbf c$ un cardinal fortement inaccessible non dénombrable, et $X$ un ensemble de type strict $\mathbf c$. Alors il existe un cardinal $\mathbf d < \mathbf c$ tel que $X$ soit de type $\mathbf d$.}

En effet, pour tout entier $n$, notons $\mathbf d_n$ le cardinal de l'ensemble $X_n$ des composants d'ordre $n$ de $X$. On a $\mathbf d_0=1$, $\mathbf d_1=\card(X) < \mathbf c$, et, comme $X_n=\bigcup_{y\in X_{n-1}} y$, on en déduit, par récurrence sur $n$ et usage de (FI.2), qu'on a $\mathbf d_n < \mathbf c$ pour tout $n$. Alors les $\mathbf d_n$ sont majorés par le cardinal $\mathbf d=\sum_{j\geq 0}\mathbf d_j$, qui est $ < \mathbf c$ par (FI.2) et l'hypothèse de non dénombrabilité de $\mathbf c$. Alors, si $Y$ est un composant d'ordre $n$ de $X$, on a $\card(Y)\leq \card(X_{n+1})\leq\mathbf d$. cqfd

\nn\textbf{Proposition~10.} \emph{Soient $U$ un univers et $\mathbf c$ un cardinal fortement inaccessible. Alors l'ensemble $U'$ des $x\in U$ qui sont de type strict $\mathbf c$ est un univers.}

En effet, si $x$, $y\in U'$ et si $z\in x$, on a évidemment $z\in U'$ et $\{x,y\}\in U'$. Si $x\in U'$, on a $\card(x) < \mathbf c$, d'où $\card(\mathcal{P}(x)) < \mathbf c$ par (FI.1); comme un composant de $\mathcal{P}(x)$ est, ou bien $\mathcal{P}(x)$, ou bien une partie de $x$, ou bien un composant de $x$, on a $\mathcal{P}(x)\in U'$. Enfin si $(x_{\alpha})_{\alpha\in I}$ est une famille d'éléments de $U'$ telle que $I\in U'$, on a $\card(\bigcup_{\alpha}x_{\alpha}) < \mathbf c$ par (FI.2); comme tout composant d'ordre $>0$ de $\bigcup_{\alpha}x_{\alpha}$ est un composant d'un $x_{\alpha}$, on a bien $\bigcup_{\alpha}x_{\alpha}\in U'$. cqfd

\nn\textbf{Remarque sur le cardinal} $\mathbf c(U)$. Soient $U$ un univers et $\mathbf c(U)$ le cardinal défini par (1). ie.
$$
\mathbf c(U)=\sup_{x\in U}\card(x).
$$
On a évidemment $\mathbf c(U)\leq \card(U)$, car, rappelons-le, $x\in U$ implique $x\subset U$. Mais l'égalité $\mathbf c(U)=\card(U)$ n'est pas toujours vraie. Par exemple soit $(x_{\alpha})_{\alpha\in I}$ une famille \emph{non dénombrable} de symboles avec les axiomes:
$$
\text{\og}x_{\alpha}=\{x_{\alpha}\}\text{ pout tout }\alpha\in I\text{\fg},\quad\text{\og}x_{\alpha}\neq x_{\beta}\text{ si }\alpha\neq\beta\text{\fg}
$$
(autrement dit $x_{\alpha}$ est un ensemble à un seul élément, à savoir lui-même). Formons, comme dans l'ex.~2 du $\rm n^\circ$~1, les  \og mots significatifs\fg\ formés avec les symboles $\varnothing$, $x_{\alpha}$, $\{,\}$ etc. Chacun désigne un ensemble \emph{fini}. Soit $U$ l'ensemble de ceux-ci. On vérifie aisément que les conditions de la déf.~1 sont satisfaites, de sorte que $U$ est un \emph{univers}. Comme les mots significatifs sont des suites finies d'éléments d'un ensemble de cardinal $\card(I)+4=\card(I)$; on a $\card(U)=\card(I)$ (chap.~III; en fait $\card(U)\geq \card(I)$ suffit, et c'est évident). D'autre part, $\mathbf c(U)$ est le cardinal dénombrable, d'où $\mathbf c(U) < \card(U)$.

L'inégalité stricte $\mathbf c(U) < \card(U)$ tient à ce qu'on a introduit ici des ensembles $x_{\alpha}$ tels que $x_{\alpha}\in x_{\alpha}$. Si on interdit des horreurs de ce genre, on obtient $\card(U)=\mathbf c(U)$ pour tout univers $U$, ainsi que d'autres fort jolis résultats \og ne pouvant servir à rien\fg. C'est ce qu'on va faire au numéro suivant.

\section{Ensembles et univers artiniens}

\nn\textbf{Définition 5.} \emph{On dit qu'un ensemble $x$ est artinien s'il n'existe aucune suite infinie $(x_n)_{n\geq 0}$ telle que $x_0=x$ et que $x_{n+1}\in x_n$ pour tout $n\geq 0$.}

\nn\textbf{Exemples.} Les ensembles $\varnothing$, $\{\varnothing\}$, $\{\varnothing,\{\varnothing\}\}$, plus généralement les éléments de l'univers $U_1$ du $\rm n^\circ$~1, sont artiniens.

En termes imagés, si $x$ est artinien, et si on prend un élément $x_1$ de $x$, puis un élément $x_2$ de $x_1$, etc, le processus doit s'arrêter, et on arrive à un composant $x_n$ de $x$ qui est \emph{vide}. Autrement dit les ensembles artiniens sont \og construits à partir de $\varnothing$\fg; ceci sera précisé plus tard (cf. (*) dans la démonstration du th.~2).

Les ensembles artiniens jouissent évidemment des propriétés suivantes:

\nn(AR.I) Si $x$ est artinien, toute partie de $x$ et tout composant de $x$ sont artiniens. Pour que $x$ soit artinien, il faut et il suffit que tout élément de $x$ soit artinien.

\nn(AR.II) Si $x$ et $y$ sont artiniens, alors $\{x,y\}$ est
artinien.

\nn(AR.III) Si $x$ est artinien, $\mathcal{P}(x)$ est artinien.

\nn(AR.IV) Toute réunion d'ensembles artiniens est un ensemble artinien.

Ces propriétés montrent aussitôt qu'on a la

\nn\textbf{Proposition~11.} \emph{Si $U$ est un univers, l'ensemble des éléments artiniens de $U$ est un univers, nécessairement artinien.}

\nn \textbf{Corollaire.} \emph{Si $x$ est un ensemble artinien, il existe un univers artinien $U$ tel que $x\in U$.}

En effet, par l'axiome (A.6), $x$ est élément d'un univers $V$; on prend pour $U$ l'ensemble des éléments artiniens de $V$.

La proposition suivante est encore moins utile que le reste du $\rm n^\circ$:

\nn\textbf{Proposition~12.} \emph{Soit $A$ un ensemble artinien. Alors}:

\nn a) \emph{Pour tout $x\in A$, on a $x\not\in x$};

\nn b) \emph{Si $x$, $y\in A$, on ne peut avoir à la fois $x\in y$ et $y\in x$};

\nn c) \emph{Pour tout $x\in A$, la relation \og$y$ est un composant de $z$\fg\ entre composants $y,z$ de $x$ est une relation d'ordre};

\nn d) \emph{Pour tout élément non vide $x$ de $A$, il existe $y\in x$ tel que $x\cap y=\varnothing$.}

En effet la négation de a) (resp. de b)) entraîne l'existence d'une suite infinie $(x_n)_{n\geq 0}$ contredisant le déf. 5, à savoir $(x,x,x,\ldots)$ (resp. $(x,y,x,y,x,y,\ldots)$). La négation de c) veut dire qu'il existe $x\in A$, des composants $y$, $z$ de $x$ distincts, et des suites d'appartenances\footnote{In [TV] the comma in the next display is replaced with the conjonction ``et''.}:
$$
y\in y_1\in\cdots\in y_q\in z,\quad z\in z_1\in\cdots\in z_r\in y;
$$
d'où, comme dans b), une suite infinie $(x_n)_{n\geq 0}$ contredisant la\footnote{In [TV] it is written "le déf." instead of "la déf.".} déf. 5. Enfin, si d) est fausse, il existe un élément non vide $x$ de $A$ tel que $x\cap y\neq \varnothing$ pour tout $y\in x$; on pose $x_0=x$, et on prend pour $x_1$ un élément de $x$; comme $x\cap x_1\neq \varnothing$, on prend pour $x_2$ un élément de $x\cap x_1$ et caetera; plus formellement on définit par récurrence une suite infinie $(x_n)_{n\geq 1}$ d'éléments de $x$ au moyen de $x_1\in x$, $x_{n+1}\in x_n\cap x$ pour $n\geq 1$; alors la suite $(x_n)_{n\geq 0}$ contredit la déf. 5.

\nn\textbf{Remarque.} Soit $B$ un ensemble. Pour que $B$ soit artinien, il faut et il suffit que tout ensemble $A$ d'ensembles de composants de $B$ satisfasse à la condition d) de la prop.~12. En effet la nécessité résulte de (AR.I) et de la prop.~12. Réciproquement, si $B$ n'est pas artinien, il existe une suite infinie $(x_n)_{n\geq0}$ avec $x_0=B$ et $x_{n+1}\in x_n$ pour tout $n\geq 0$; on prend alors pour $A$ la partie réduite à l'ensemble $X$ des $x_n$; ainsi $A$ contient un élément non vide $X$ tel que, pour tout élément $y$ de $X$, on ait $y\cap X\neq\varnothing$ (en effet $y$ est de la forme $x_n$, et on a $x_{n+1}\in y\cap X$).

\nn\textbf{Théorème~2.} \emph{Soit $\mathbf c$ un cardinal infini. Alors:}

\nn a) \emph{La relation \og$x$ est un ensemble artinien de type $\mathbf c$ (resp. de type strict $\mathbf c$)\fg\ est collectivisante par rapport à $x$; l'ensemble $A_{\mathbf c}$ des ensembles artiniens de type $\mathbf c$ a pour cardinal $2^{\mathbf c}$.}

\nn b) \emph{Si $\mathbf c$ est fortement inaccessible, l'ensemble $U_{\mathbf c}$ des ensembles artiniens de type strict $\mathbf c$ est un univers de cardinal $\mathbf c$; le cardinal $\mathbf c(U_{\mathbf c})=\sup_{x\in U_{\mathbf c}}\card(x)$ est $\mathbf c$.}

\nn c) \emph{Si un univers $U$ admet un élément de cardinal $\mathbf c$, tout ensemble artinien de type $\mathbf c$ appartient à $U$ (autrement dit $A_{\mathbf c}\subset U$).}

\nn c') \emph{Si un univers $U$ est non vide, tout ensemble artinien de type fini est élément de $U$.}

Avant de démontrer le th.~2, déduisons en quelques corollaires illuminants:

\nn \textbf{Corollaire~1.} Si un univers $U$ est artinien, alors $\card(U)$ est fortement inaccessible, et $U$ est l'ensemble des ensembles artiniens de type strict $\card(U)$.

En effet posons $\mathbf c(U)=\sup_{x\in U}\card(x)$. C'est un cardinal fortement inaccessible (début du $\rm n^\circ$~5). Supposons le d'abord non dénombrable; alors, pour tout cardinal infini $\mathbf c < \mathbf c(U)$, tout ensemble artinien de type $\mathbf c$ est élément de $U$ par c); donc tout ensemble artinien de type strict $\mathbf c(U)$ est élément de $U$ par le lemme du $\rm n^\circ$~5. Cette dernière assertion reste valable si $\mathbf c(U)$ est dénombrable par $\rm c'$). Inversement, d'après (U.I), tout ensemble de $U$ est de type strict $\mathbf c(U)$. Donc $U$ est l'univers $U_{\mathbf c(U)}$ de b), d'où $\mathbf c(U)=\card(U)$ par b).

Il résulte du cor.~1 qu'un\footnote{In [TV] it is written ``qu'on univers'' instead of ``qu'un univers''.} univers artinien est \emph{déterminé} de façon unique par son cardinal (d'ailleurs fortement inaccessible). On a donc une \og correspondance biunivoque\fg\ entre univers artiniens et cardinaux fortement inaccessibles. En particulier:

\nn\textbf{Corollaire~2.}  La relation d'inclusion $U\subset U'$ entre univers artiniens est une relation de bon ordre.

En effet la relation $\mathbf c\leq\mathbf c'$ entre cardinaux est une relation de bon ordre (chap.~III) et, avec les notations du b) du th.~2, les relations $\mathbf c\leq\mathbf c'$ et $U_{\mathbf c}\subset U_{\mathbf c'}$ sont équivalentes.

Notons que le th.~2, b) donne une seconde démonstration du th.~1 ($\rm n^\circ$~5).

Passons à la démonstration du th.~2. \'Etant donné un ensemble $A$, nous appellerons \emph{chaîne} de $A$ toute suite finie $(x_j)_{j=0,\ldots,n}$ telle que $x_0=A$ et que $x_{j+1}\in x_j$ pour $j=0,\ldots,n-1$. Les chaînes de $A$ forment un \emph{ensemble}, d'après le schéma de sélection-réunion; notons le $G(A)$. \'Etant donnée une chaîne $X=(x_j)_{j=0,\ldots,n}$, les chaînes de la forme $(x_i)_{i=0,\ldots,q}$ avec $q\leq n$ seront dites plus petites que $X$; on obtient ainsi, sur $G(A)$, une structure d'\emph{ensemble ordonné}. Pour que $A$ soit artinien, il faut et il suffit que $G(A)$ soit un ensemble ordonné \og\emph{noethérien}\fg\ (c.à.d. satisfaisant aux conditions équivalentes du chap. III, $\S$ 6, $\rm n^\circ$~5).

Nous allons montrer que:

(*) \emph{Si $A$ est artinien, il est déterminé de façon unique par la classe d'isomorphisme de l'ensemble ordonné $G(A)$.}

En effet, étant donnés un ensemble ordonné $G$ et un élément $g\in G$, nous noterons $S(g)$ l'ensemble des $g'\in G$ tels que $g < g'$ et que $g\leq h\leq g'$ implique $h=g$ ou $h=g'$ (autrement dit l'ensemble des \og successeurs immédiats\fg\ de $g$). Considérons l'application $\theta$ qui, à toute chaîne $X=(x_0,\ldots,x_n)$ de $G(A)$ fait correspondre l'ensemble $x_n$; on a alors, pour tout $X\in G(A)$:
$$
\theta(X)=\{\theta(X')| X'\in S(X)\}.
$$
Comme $A$ est l'image par $\theta$ du plus petit élément $X_0=(A)$ de $G(A)$, il va nous suffire de montrer que $\theta$ est uniquement déterminée par la classe d'isomorphisme de l'ensemble ordonné $G(A)$. Or ceci résulte du lemme suivant:

\nn\textbf{Lemme~1.} \emph{Soient $G$ un ensemble ordonné noethérien et $\varphi$ une application de $G$ telle que, pour tout $g\in G$, on ait}:
\begin{equation*}
\varphi(g)=\{\varphi(g')| g'\in S(g)\}.\tag{1}
\end{equation*}
\emph{Alors $\varphi$ est déterminée de façon unique. De plus, si $U$ est un univers contenant un élément équipotent à $G$, $\varphi$ prend ses valeurs dans $U$.}

L'hypothèse implique qu'on a $\varphi(g)=\varnothing$ si $g$ est un élément maximal de $G$.

Soient, en effet, $\varphi$ et $\varphi'$ deux applications telles que (1) soit vraie; si $\varphi\neq\varphi'$, l'ensemble des $g\in G$ tels que $\varphi(g)\neq\varphi'(g)$ est non-vide, donc admet un élément \emph{maximal} $h$ car $G$ est noethérien; on a alors $\varphi(g)=\varphi'(g)$ pour tout $g>h$, en particulier pour tout \og successeur\fg\ $g\in S(h)$; d'où $\varphi(h)=\varphi'(h)$ par (1), ce qui est une contradiction; on a donc bien $\varphi=\varphi'$. Soit maintenant $U$ un univers contenant un élément $x$ équipotent à $G$; montrons que $\varphi$ prend ses valeurs dans $U$; sinon soit $h$ un élément maximal parmi les $g\in G$ tels que $\varphi(g)\not\in U$; on a $\varphi(g')\in U$ pour tout $g'\in S(g)$ de sorte que
$$
\varphi(h)=\{\varphi(g')| g'\in S(h)\}
$$
est une partie de $U$; or, comme son cardinal est inférieur à $\card(G)$ donc au cardinal d'un élément de $U$, on a $\varphi(h)\in U$ ($\rm n^\circ$~1, prop.~7); cette contradiction montre que $\varphi$ prend ses valeurs dans $U$.

Ceci étant, démontrons le a) du th.~2. On peut se borner à l'assertion non-respée\footnote{The author probably means this: Recall that Part a) of Theorem~2 says: ``La relation \og$x$ est un ensemble artinien de type $\mathbf c$ (resp. de type strict $\mathbf c$)\fg\ est collectivisante par rapport à $x$; l'ensemble $A_{\mathbf c}$ des ensembles artiniens de type $\mathbf c$ a pour cardinal $2^{\mathbf c}$.'' Then, l'assertion non-respée is: ``la relation \og$x$ est un ensemble artinien de type $\mathbf c$\fg\ est collectivisante par rapport à $x$; l'ensemble $A_{\mathbf c}$ des ensembles artiniens de type $\mathbf c$ a pour cardinal $2^{\mathbf c}$''. (L'assertion respée would be ``la relation \og$x$ est un ensemble artinien de type \emph{strict} $\mathbf c$\fg\ est collectivisante par rapport à $x$; l'ensemble $A_{\mathbf c}$ des ensembles artiniens de type $\mathbf c$ a pour cardinal $2^{\mathbf c}$.'')}, car l'autre en découle aussitôt. Soit $\mathbf c$ un cardinal \emph{infini}. Si $A$ est un ensemble de type $\mathbf c$ on a:
\begin{equation*}
\card(G(A))\leq\mathbf c.\tag{2}
\end{equation*}
En effet, si on note $A_n$ l'ensemble des composants d'ordre $n$ de $A$, on a $\card(A_1)\leq \mathbf c$ et $\card(A_{n+1})\leq\mathbf c\cdot \card(A_n)$, d'où $\card(A_n)\leq\mathbf c^n$ par récurrence; or $\mathbf c^n=\mathbf c$ car $\mathbf c$ est infini (chap. III); d'où $\card(\bigcup^\infty_{n=0} A_n)\leq\card(\NN)\cdot \mathbf c=\mathbf c$ (chap. III); or $G(A)$ est un ensemble de suites \emph{finies} d'éléments de
$\bigcup_n A_n$, de sorte qu'on a bien l'inégalité (2) (chap. III). Ceci étant, si $E$ est un ensemble de cardinal $\mathbf c$, la donnée d'une structure d'ordre sur une partie $E'$ de $E$ équivaut à la donnée de la partie de $E\times E$ formée des $(x,y)$ tels que $x\in E'$, $y\in E'$ et $x\leq y$. Donc, en vertu de (2), les classes d'isomorphisme des ensembles ordonnés $G(A)$ (où $A$ est de type $\mathbf c$) forment un ensemble $\mathfrak{F}_{\mathbf c}$, et on a $\card(\mathfrak{F}_{\mathbf c})\leq\card(\mathcal{P}(E\times E))=2^{\mathbf c}$. Soit $\mathfrak{F}'_{\mathbf c}$ la partie de $\mathfrak{F}_{\mathbf c}$ formée des classes d'ensembles ordonnés noethériens ayant un plus petit élément; si, à tout $G\in\mathfrak{F}'_{\mathbf c}$, on fait correspondre la valeur de la fonction $\varphi$ du lemme~2 au plus petit élément de $G$, on obtient une application $\theta$ de $\mathfrak{F}'_{\mathbf c}$ dont l'image contient \emph{tous} les ensembles artiniens de type $\mathbf c$. Ces derniers forment donc bien un ensemble $A_{\mathbf c}$, et on a $\card(A_{\mathbf c})\leq\card(\mathfrak{F}'_{\mathbf c})\leq\card(\mathfrak{F}_{\mathbf c})\leq 2^{\mathbf c}$.

Reste à voir qu'on a $\card(A_{\mathbf c})=2^{\mathbf c}$. Pour cela il suffit de voir qu'il existe un ensemble artinien $B$ de type $\mathbf c$ et de cardinal $\mathbf c$, car les parties de $B$ seront alors des éléments de $A_{\mathbf c}$. Cette existence résulte du lemme suivant:

\nn\textbf{Lemme~2.} \emph{Pour tout cardinal $\mathbf c$, il existe un ensemble artinien $B$ de type $\mathbf c$ et de cardinal $\mathbf c$.}

On procède par induction transfinie sur $\mathbf c$. Pour $\mathbf c=0$ on n'a pas le choix, et on prend $B=\varnothing$. Si $\mathbf c$ à un prédécesseur $\mathbf c'$, soit $B'$ un ensemble artinien de type $\mathbf c'$ et de cardinal $\mathbf c'$; on a alors $\mathbf c\leq 2^{\mathbf c'}$, de sorte qu'il existe une partie $B$ de $\mathcal{P}(B')$ de cardinal $\mathbf c$; les éléments de $B$ sont des parties de $B'$ et ont donc des cardinaux $\leq\mathbf c'\leq\mathbf c$; les composants d'ordre supérieur de $B$ sont des composants de $B'$, et ont donc aussi des cardinaux $\leq\mathbf c'\leq\mathbf c$. Enfin, si $\mathbf c$ n'a pas de prédécesseur, on choisit, pour tout cardinal $\mathbf c_{\lambda} < \mathbf c$, un ensemble artinien $B_{\lambda}$ de type $\mathbf c_{\lambda}$ et de cardinal $\mathbf c_{\lambda}$; alors $B=\bigcup _{\lambda}B_{\lambda}$ répond à la question. Ceci démontre le lemme~2, et achève la démonstration de la partie a).

Passons à b). Soit $\mathbf c$ un cardinal fortement inaccessible. On sait déjà, par a), que les ensembles artiniens de type strict $\mathbf c$ forment un ensemble $U_{\mathbf c}$. Le fait que $U_{\mathbf c}$ est un univers résulte aussitôt des propriétés (AR.I) à (AR.IV) des ensembles artiniens (début du $\rm n^\circ$), et de majorations de cardinaux analogues à celles de la prop.~10 du $\rm n^\circ$~5. La relation $\sup_{x\in U_{\mathbf c}} \card(x) = \mathbf c$ résulte du lemme~2,  appliqué aux cardinaux $ < \mathbf c$. Enfin, pour montrer que $\card(U_{\mathbf c}) = \mathbf c$, supposons d'abord $\mathbf c$ 
\emph{non dénombrable}; d'après le lemme du $\rm n^\circ$~5, $U_{\mathbf c}$ est la réunion $\bigcup_{\mathbf d < \mathbf c} A_{\mathbf d}$, où $A_{\mathbf d}$ désigne l'ensemble des ensembles artiniens de type $\mathbf d$; or on a $\card(A_{\mathbf d})= 2^{\mathbf d} < \mathbf c$ (par a)); d'autre part l'ensemble des cardinaux $\mathbf d < \mathbf c$ a un cardinal $\leq \mathbf c$ (chap. III); d'où $\card(U_{\mathbf c})\leq \mathbf c \cdot \mathbf c=\mathbf c$, et aussi $\card(U_{\mathbf c})\geq \mathbf c$ car $\card(U_{\mathbf c})\geq \card(A_{\mathbf d})=2^{\mathbf d}$ pour tout $\mathbf d < \mathbf c$.

Le cas $\mathbf c=0$ étant trivial, reste le cas où $\mathbf c$ est le \emph{cardinal infini dénombrable}. Dans ce cas $U_{\mathbf c}$ est l'ensemble des ensembles artiniens de type fini (ie. finis ainsi que tout leurs composants), et on utilise un joli résultat de nature combinatoire.

\nn\textbf{Lemme 3.} (D. König ?). \emph{Considérons deux suites infinies $(E_n)_{n\geq1}$, $(f_n)_{n \geq 1}$ d'ensembles finis $E_n$ et d'applications $f_n:E_{n+1} \rightarrow E_n$. S'il n'existe aucune suite infinie $(x_n)_{n\geq 1}$ telle que $x_n\in E_n$ et que $f_n(x_{n+1})=x_n$ pour tout $n$, alors $E_n$ est vide pour $n$ assez grand.}

Autrement dit, si toutes les suites $(x_n)$ telles que $f_n(x_{n+1})=x_n$ sont \emph{finies}, leurs longueurs sont \emph{bornées}. Ça peut s'exprimer en termes de limites projectives: une limite projective d'ensembles finis non vides est non vide (cf. Top. Géné., Chap.~I, 2e éd. $\S 9,\ \rm n^\circ$~6, prop.~8, $2^\circ$)).

Raisonnons, en effet, par l'absurde. S'il existe des $E_n$ non vides pour $n$ arbitrairement grand, aucun $E_n$ n'est vide (car $E_n=\varnothing$ entraîne $E_{n+1}=\varnothing$ vu l'existence de $f_n:E_{n+1}\rightarrow E_n$). Appelons \og cohérentes\fg\ les suites finies $(x_j)_{1\leq j\leq n}$ telles que $f_j(x_{j+1})=x_j$ pour $j=1,\ldots,n-1$. Démontrons, par récurrence sur $n$, l'existence d'une suite cohérente $(a_1,\ldots,a_n)$, $(a_i\in E_i)$, qui, pour tout $q\geq n$, peut être prolongée en une suite cohérente $(a_1,\ldots,a_n,x_{n+1},\ldots,x_q)$ de longueur $q$. C'est évident pour $n=0$, car aucun $E_q$ n'est vide. Passons de $n$ à $n+1$. Si, pour tout $x\in f^{-1}_n(\{a_n\}) \subset E_{n+1}$, toutes les\footnote{In [OV] and [TV] ``les'' is written ``des''.} suites cohérentes prolongeant $(a_1,\ldots,a_n,x)$ avaient des longueurs bornées par un entier $q(x)$, alors toutes les suites cohérentes prolongeant $(a_1,\ldots,a_n)$ seraient de longueurs bornées (par $\sup_x q(x)$) car $E_{n+1}$ est fini; il existe donc $a_{n+1} \in f^{-1}_n(\{a_n\})$ tel que la suite cohérente $(a_1, \ldots, a_n, a_{n+1})$ admette des prolongements de longueur arbitraire. Ceci étant on a une suite infinie $(a_n)_{n\geq 1}$ qui contredit l'hypothèse.

Il résulte du lemme 3 que si $A$ est un ensemble artinien de type fini, alors l'ensemble ordonné $G(A)$ de ses chaînes est \emph{fini}: on prend, en effet, pour $E_n$ l'ensemble des chaînes $(x_n\in x_{n-1} \in \cdots \in x_1\in A)$ à $n+1$ termes (qui est fini car les composants de $A$ d'ordre $\leq n+1$ sont en nombre fini et sont tous\footnote{In [TV] it is written ``tout'' instead of ``tous''.} finis), et pour $f_n$ l'application $(x_{n+1}\in x_n \in x_{n+1} \in \cdots)\mapsto (x_n\in x_{n-1}\in\cdots)$. Or l'ensemble des classes d'isomorphisme d'ensembles ordonnés finis est \emph{dénombrable}: en effet, la donnée d'une structure d'ordre sur une partie finie de $\NN$ équivaut à celle de son graphe, qui est une partie finie de $\NN \times \NN$; d'autre part l'ensemble des parties finies d'un ensemble dénombrable est dénombrable (chap. III). Il résulte de (*) et du lemme~1, que l'ensemble $\mathfrak{a}$ des ensembles artiniens de type fini est dénombrable. Il est infini par le lemme~2 (ou, plus simplement, par usage de la suite $(z_n)_{n\geq 0} $ définie au moyen de $z_0=\varnothing$, $z_{n+1}=\{z_n\}$). Ceci termine la démonstration de b).

\nn\textbf{Remarque.} On vient de démontrer que, si $A$ est un ensemble artinien de type fini, il n'a qu'un nombre fini de composants. Il existe donc un entier $n$ tel que $A$ soit de type $n$.

Passons à la démonstration de c). Soient $\mathbf c$ un cardinal infini, $U$ un univers admettant un élément de cardinal $\mathbf c$, et $A$ un ensemble artinien de type $\mathbf c$. Alors l'ensemble ordonné $G(A)$ a un cardinal $\leq \mathbf c$ (formule (2) ci-dessus). La seconde assertion du lemme~1, appliquée à $G(A)$, montre que l'application $(x_0,\ldots, x_n)\mapsto x_n$ de $G(A)$ prend ses valeurs dans $U$. Autrement dit tout les composants de $A$ sont éléments de $U$. Ceci démontre c). La démonstration de $\rm c'$) est analogue: si $A$ est un ensemble artinien de type fini, on vient de voir que $G(A)$ est fini; comme $U$ est non-vide, il contient un élément équipotent à $G(A)$ par le cor. à la prop.~7 ($\rm n^\circ$~1). cqfd

\section{Remarques métamathématiques vaseuses}

\nn a) L'axiome \og\emph{tout ensemble est artinien}\fg\ est \emph{inoffensif}. En effet, si on a un modèle $M$ de la Théorie des Ensembles, l'ensemble $M'$ des éléments artiniens de $M$ est aussi un modèle (cf. prop.~11).

\nn b) L'axiome (A.6) des univers (et l'axiome équivalent (A'.6) des cardinaux fortement inaccessibles) est \emph{indépendant} du reste de la Théorie des Ensembles. En effet soit $\mathbf c$ le premier cardinal fortement inaccessible non dénombrable. L'univers $U_{\mathbf c}$ des ensembles artiniens de type strict $\mathbf c$ (th.~2, b)) est un modèle de la Théorie des Ensembles sans (A.6): on appelle \og ensembles\fg\ les éléments de $U_{\mathbf c}$, la relation d'appartenance est la restriction à $U_{\mathbf c}$ de l'ordinaire, etc. Les \og univers\fg\ du modèle sont donc les univers ordinaires qui sont éléments de $U_{\mathbf c}$. Or on a vu que les seuls univers qui sont éléments de $U_{\mathbf c}$ sont les deux pequeños $U_0 = \varnothing$ et $U_1$. Ainsi $U_1$ est un \og ensemble\fg\ qui n'est élément d'aucun \og univers\fg. On a donc un modèle de la Théorie des Ensembles où (A.6) est faux.

\nn c) Bourbaki a été trop prudent en se contentant de \og présumer\fg\ que l'\emph{axiome de l'infini} (A.5) est \emph{indépendant} des axiomes et schémas précédents. Il l'est effectivement, car l'univers dénombrable $U_1$ des ensembles artiniens de type fini est un modèle où (A.5) est faux, et où les axiomes et schémas précédents sont vrais.

\nn d) Il serait très intéressant de démontrer que l'axiome (A.6) des univers est inoffensif. Ça paraît difficile et c'est même indémontrable, dit Paul Cohen.

L'adjectif \og vaseuses\fg\ dans le titre veut dire qu'on ne s'est pas donné la peine, en construisant des modèles, de canuler le symbole $\tau$ de sorte qu'il n'en fasse pas sortir. La clef de ça, si on considère un modèle $M$, est de remplacer le $\tau_x\left(\underline{R}(x)\right)$ ordinaire par:
$$
\tau_x\left(\underline{R}(x)\text{ et } x\in M\right).
$$
Encore faut-il vérifier que ça transforme bien les quantificateurs ordinaires en les quantificateurs autrefois dits \og typiques\fg:
$$
(\forall x \in M)\text{ et }(\exists x \in M).
$$

\section{Exercices}

\nn1) Soit $n$ un entier $\geq 1$. Montrer que l'ensemble des ensembles artiniens de type $n$ est infini (les ensembles $z_0 = \varnothing$, $z_1=\{\varnothing\}$, $z_{q+1} = \{z_q\}$ sont de type~1).

\nn2) Appelons \emph{hauteur} d'un ensemble $A$ la borne supérieure (finie ou infinie) des entiers $n$ tels qu'il existe une suite $x_n\in x_{n-1}\in \cdots \in x_0 = A$\footnote{In [OV] and [TV] the "$A$" is missing.}. Montrer que les ensembles de hauteur $\leq n$ sont finis et forment un ensemble fini, dont le cardinal $p_n$ se calcule au moyen de $p_0=1$, $p_{n+1}=2^{p_n}$ (procéder par récurrence sur $n$, en notant que les éléments d'un ensemble $A$ de hauteur $\leq n$ sont des ensembles de hauteur $\leq n-1$).

\nn3) Soit $G$ un ensemble ordonné noethérien admettant un plus petit élément $g_0$ et tel que pour tout $h\in G$, l'ensemble de $g\leq h$ sont fini et totalement ordonné.

a) Montrer que tout élément $h\neq g_0$ de $G$ admet un prédécesseur et un seul.

b) Soit $\varphi$ l'application de $G$ définie dans le lemme~1 (ie. $\varphi(g)$ est, l'ensemble des $\varphi(g')$ où $g'$ parcourt l'ensemble des successeurs de $g$). On pose $A=\varphi(g_0)$. Montrer que $A$ est un ensemble artinien. Pour $g\in G$, soit $g_0 < g_1 < \cdots < g_n=g$ la suite des éléments $\leq g$; posons $f(g)=(\varphi(g_0),\varphi(g_1),\ldots,\varphi(g_n))$; montrer que $f$ est une application croissante et surjective de $G$ sur l'ensemble ordonné $G(A)$ du texte. Montrer que, si $f$ est \emph{injective}, c'est un \emph{isomorphisme} de $G$ sur $G(A)$.

c) Pour $g\in G$, soit $M_g$ l'ensemble des majorants de $g$. On suppose que, pour tout couple d'éléments distincts $g$, $g'$ ayant même prédécesseur $p$, les ensembles ordonnés $M_g$ et $M_{g'}$ sont non isomorphes. Montrer que l'application $f$ de b) est alors un \emph{isomorphisme} de $G$ sur $G(A)$. [Si $f(g)=f(g')$ avec $g\neq g'$ et si $g_0 < g_1 < \cdots < g_n=g$ et $g'_0 < g'_1 < \cdots  < g'_{n'}=g'$ sont la suite des éléments $\leq g$ et celles des éléments $\leq g'$, montrer que $n=n'$, et qu'on peut supposer que $g$ et $g'$ ont le même prédécesseur $p$; considérer alors l'ensemble des $p\in G$ tels qu'il existe deux successeurs distincts $g$, $g'$ de $p$ tels que $f(g)=f(g')$, un élément maximal $q$ de cet ensemble, et deux successeurs distincts $h$, $h'$ de $q$ tels que $f(h)=f(h')$; noter que les restrictions de $f$ à $M_h$ et à $M_{h'}$ sont injectives, donc (par b)) sont des isomorphismes de $M_h$ sur $G(\varphi(h))$ et de $M_{h'}$ sur $G(\varphi(h'))$; déduire de l'hypothèse de non-isomorphisme de $M_h$ et $M_{h'}$ qu'on a $\varphi(h)\neq \varphi(h')$,\footnote{In [TV] there is a period instead of the comma.} ce qui contredit $f(h)=f(h')$.]

\textbf{N.B.} L'exercice 3) donne des renseignements très précis sur la manière dont sont \og fabriqués\fg\ les ensembles artiniens. On savait déjà qu'un tel ensemble $A$ est déterminé par la classe d'isomorphisme de l'ensemble ordonné $G(A)$ (lemme~1). On sait maintenant \emph{caractériser} les ensembles ordonnés $G$ isomorphes à des $G(A)$: 

$\alpha)$ $G$ est noethérien et admet un plus petit élément;

$\beta)$ Pour tout $g\in G$, l'ensemble des $h \leq g$ est totalement ordonné et fini (d'où l'existence et l'unicité du prédécesseur de $g$);

$\gamma)$ Si $g$, $g'$ sont des éléments distincts ayant même prédécesseur, l'ensemble $M_g$ des majorants de $g$ et celui $M_{g'}$ des majorants de $g$ ne sont pas isomorphes.

\nn4) Soit $(A_n)_{n\geq 0}$ la suite des \emph{ordinaux} finis, définie par $A_0=\varnothing$, $A_{n+1}=A_n \cup \{A_n\}$. Montrer que l'ensemble ordonné $G(A_n)$ a $2^n$ éléments, et que le nombre de ses éléments de hauteur $q$ (au sens de l'exerc.~2)) est $\binom{n}{q}$.\bigskip

\centerline*

This text is available at 

\nn my site: \href{http://iecl.univ-lorraine.fr/~Pierre-Yves.Gaillard/DIVERS/Retyping/}{http://iecl.univ-lorraine.fr/$\sim$Pierre-Yves.Gaillard/DIVERS/Retyping/}

\nn box.com: \href{https://app.box.com/s/il39ow5efnusznnbvik0}{https://app.box.com/s/il39ow5efnusznnbvik0}

\nn dropbox: \href{http://goo.gl/G2oSgc}{http://goo.gl/G2oSgc}

\nn github: \href{https://github.com/Pierre-Yves-Gaillard/retyping}{https://github.com/Pierre-Yves-Gaillard/retyping}

\nn google: \href{http://goo.gl/J8lzRV}{http://goo.gl/J8lzRV}

\nn mediafire: \href{https://www.mediafire.com/folder/q5jv3o9dg0vzz/Retyping}{https://www.mediafire.com/folder/q5jv3o9dg0vzz/Retyping}

\nn mega: \href{https://mega.co.nz/#F!qE5j1ZQB!Vs6w2mIoggIxRQSinHxZZw}{https://mega.co.nz/\#F!qE5j1ZQB!Vs6w2mIoggIxRQSinHxZZw}
\end{document}
